
\begin{subsection}{Registración de un usuario}

El usuario se conecta a internet con algún dispositivo(pc, tablet, smartphone) y accede a la página
web del sistema de bicicletas de Mar Chiquita. Como el usuario no se encuentra registrado, necesita 
registrarse para poder acceder al servicio. Para lograrlo deberá ingresar su número de D.N.I, si dicho
número no fue registrado, el sistema le pedirá que ingrese una contraseña para proteger su cuenta. En cambio
si dicho número ya estuviese asignado el sistema le informara al usuario que deberá acercarse a cualquier
terminal del servicio para resolver su situación.

Si la registración fue exitosa, el usuario podrá ingresar de forma opcional la siguiente información:
	\begin{itemize}
	\item Nombre y apellido.
	\item Género.
	\item Dirección donde reside.
	\item Edad.
	\item Profesión.
	\item Estado civil.
	\item Correo electrónico.
	\end{itemize}

Luego el sistema le informará que su cuenta ha sido creada, faltando solamente el paso de verificación, donde
podrá acercarse a cualquier terminal con su D.N.I para confirmar la identidad del usuario. Luego de la 
verificación, el usuario estará en condiciones de retirar una bicicleta desde cualquier terminal.

Por otro lado si el sistema le hubiese informado que su D.N.I ya se encuentra registrado, el sistema le pedirá
que vaya a cualquier terminal para verificar su identificación y corregir su situación.
\end{subsection} 

\begin{subsection}{Retiro de una bicicleta}
Un usuario se acerca cualquier estación de su preferencia y se identifica ante el Encargado de la estación.
Dicho encargado verifica que el D.N.I se encuentre registrado, concuerde con la identidad del usuario y que no tenga ninguna penalización. Si hay stock de bicicletas registra la entrega de la misma, caso contrario le informa al Usuario que no hay bicicletas disponibles y que debe aguardar a la reposición. El sistema se encarga de pedir la reposición de las mismas.

\end{subsection} 

\begin{subsection}{Devolución de una bicicleta}
Un usuario se acerca a cualquier estación donde pretenda devolver la bicicleta pedida. El encargado de la estación
recibe el D.N.I del usuario y marca la bicicleta como recibida en el sistema(aumentado el stock disponible en la estación). En caso de que el D.N.I o el usuario no concuerden con los datos de a quien fue prestada la bicicleta el mismo será penalizado por suplantar la identidad de otro usuario. Por otro lado si el tiempo que tardó en devolver la bicicleta es superior a 1 hora, el usuario también será penalizado. Si el sistema informa que la devolución fue satisfactoria, el usuario estará habilitado a retirar otra bicicleta en cualquier estación de Mar Chiquita.

\end{subsection}

\begin{subsection}{Chequeo de stock de bicicletas en una terminal}

\end{subsection}

\begin{subsection}{Una estación se queda sin acceso a internet}

\end{subsection}

\begin{subsection}{Falla en la devolución de una bicicleta}

\end{subsection} 

\begin{subsection}{Un día en la empresa de transporte de bicicletas}

\end{subsection}

\begin{subsection}{Un día en el centro de estadísticas}

\end{subsection}