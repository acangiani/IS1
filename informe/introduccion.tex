 
%
% Primera parte.
%
\begin{subsection}{Descripción general}
Lo que sigue es una breve descripción informal de lo que hemos interpretado a partir del enunciado del trabajo. El objetivo es la construcción de un sistema informático capaz de administrar eficientemente lo que será la \emph{red de ciclovías} del distrito de \emph{Mar Chiquita}.

La red de ciclovías estará compuesta por estaciones donde se almacenan las bicletas disponibles para usuarios registrados. Cada estación tiene asignada un número determinado de bicicletas y por lo menos un empleado que las administra. El empleado atiende a los usuarios registrados que solicitan y devueven las unidades. Además, interactúa con el sistema actualizándolo con los datos de todos los movimientos de la estación que tiene a su cargo: préstamos y llegadas de unidades, penalización de usuarios, etc. El sistema también puede emitir alertas y notificaciones para estos encargados.

Los empleados de las estaciones también se ocupan de recibir y atender a los camiones destinados al desplazamiento de unidades entre estaciones.
\end{subsection}

\begin{subsection}{Lista de fenómenos y objetivos}

\begin{itemize}
\item \textbf{Fenómenos que se desprenden del enunciado:}

\begin{enumerate}
\item El usuario se registra vía Internet.
\item Si no está penalizado, el usuario registrado puede retirar solo una bicicleta en cualquiera de las estaciones.
\item Ningún usuario puede retirar más de una bicicleta.
\item El usuario debe entregar una bicicleta en uso en cualquiera de las estaciones en el plazo de una hora.
\item Los usuarios que no devuelvan las bicicletas en tiempo y forma serán penalizados.
\item Un usuario penalizado no puede retirar una bicicleta.
\item Hay dos tipos de estaciones: periféricas y centrales.
\item Todos van a la misma estación entre las 17 y 19hs.
\item Las estaciones centrales tienen la mayor demanda en determinadas franjas horarias.
\item Los usuarios no deben esperar más de 40 minutos por una bicicleta (\textbf{\emph{Objetivo blando}}).
\item Deberían haber suficientes bicicletas en las estaciones centrales para que nadie tenga que esperar demasiado (\textbf{\emph{Objetivo blando}}).
\item Solicitar reposición de bicicletas cuando una estación se queda sin bicis.
\item Conocer la disponibilidad de bicicletas.
\item Camiones transportarán bicicletas de una estación a otra.
\item Seguridad en el resguardo de identidad.
\item Almacenamiento de datos para estadísticas.
\end{enumerate}

\item \textbf{Fenómenos agregados:}

\begin{enumerate}
\item Cualquier persona puede comprobar online disponibilidad de bicicletas para cualquiera de las estaciones.
\item El usuario solicita a empleado de una estación una bicicleta.
\item El empleado consulta disponibilidad de bicicletas.
\item El empleado valida al usuario y su reputación (penalizado/no penalizado).
\item El empleado actualiza sistema (con el nuevo número de bicicletas, las que están en uso, estado del usuario).
\item El empleado entrega bicicleta al usuario.
\item Ante déficit de bicicletas en una o varias estaciones, el sistema determinará qué bicicletas desplazar de una estación a otra, según número de unidades en cada estación, bicicletas en uso y su destino, etc.
\item El sistema almacenará los datos de todos los movimientos para estadísticas.
\end{enumerate}

\item \textbf{Sobre las penalizaciones:}

\begin{enumerate}
\item \textbf{\emph{Penalización de un día}}: cuando el usuario entrega una bicicleta en un plazo de 1 a 24hs.
\item \textbf{\emph{Penalización de una semana}}: cuando el usuario entrega la bicicleta en un plazo de 1 a 7 días.
\item \textbf{\emph{Expulsion}}: cuando el usuario entrega una bicicleta pasados los 7 días, o cuando directamente la roba.
\end{enumerate}


\item{Objetivos blandos}:
Consideramos que:
\begin{itemize}
\item \emph{``Los usuarios no deben esperar más de 40 minutos por una bicicleta.''}
\item \emph{``Deberían haber suficientes bicicletas en las estaciones centrales para que nadie tenga que esperar demasiado.''}
\end{itemize}

son objetivos blandos porque son cuestiones relativas a preferencias o aspiraciones de situaciones ideales, pero que de ninguna forma el sistema puede garantizar en todo momento.

\end{itemize}
\end{subsection} 
